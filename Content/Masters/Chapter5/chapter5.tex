\chapter{Random Search}
\section{Introduction}
This chapter focuses on the performance and characteristics of Random Search. The experimental conditions are first presented before detailing  the results with some discussion on aspects of the solutions found. The degree to which wrapping occurs in the successful trials is examined and the impact on the success rate of its removal is also looked at. 

One of the key objectives of this chapter is to provide some insight into the relative difficulty of the selected problems. This is based on the assumption that Random Search as a method of sampling the search space provides some indication of the solution density associated with a particular problem.


\section{Search Strategy Options}
For Random Search, individuals of random length and random value are created and evaluated. The size of the sample is fixed at 25000, which is the number of evaluations of the objective function that have been allowed per trial to each of the search methods evaluated in this research. 

\section{Experimental Conditions}

Table~\ref{rs_param_table} shows the parameters used to configure Random Search for the trials.
For Symbolic Integration, Santa Fe Trail and Blocks a maximum genome length of 100 was used, while Symbolic Regression used a figure of 200. The selected values for maximum length were influenced by a consideration of the grammar used and the likely nature of the target expression for each of the problems. 


\begin{table}[h]
\begin{center}
\begin{tabular}{|l|l|l|l|l|}
\hline
Parameter &\multicolumn{4}{l|}{Problems}\\
\cline{2-1} \cline{3-1} \cline{4-1} \cline{5-1} 
 & Sym Int & Santa Fe & Blocks & Sym Reg  \\
\hline
Number of Trials & 1000 & 1000 & 1000 & 1000 \\
Number of Objective & & & & \\ 
Function Evaluations  & 25000 & 25000 & 25000 & 25000  \\
Initial Genome Length & 100 & 100 & 100 & 200  \\
Initial Genome Length & & & &  \\
Variation Range  & 10\% & 10\% & 10\% & 10\%  \\
Wrapping         & on   & on   & on   & on    \\
\hline
\end{tabular}
\caption{\label{rs_param_table} Parameters used to configure Random Search.}
\end{center}
\end{table}


 

\section{Results}

The results of the trials are shown in Table~\ref{rs_results_table}. Symbolic Integration proves to be the easiest problem to solve with a success rate of 40\%. The Santa Fe Trail also scores a high level of success with a figure of 30\% while Blocks is solved in 24\% of the attempts. Symbolic Regression was not solved in any of the trials.
\begin{table}[h]
\begin{center}
\begin{tabular}{|l|l|}
\hline
Problem & Successful Runs \\
\hline
Symbolic Integration & 40\% \\
Santa Fe Trail & 30\% \\
Blocks & 24\% \\
Symbolic Regression & 0\% \\
\hline
\end{tabular}
\caption{\label{rs_results_table} Results from Trials of Random Search.}
\end{center}
\end{table}

A significant aspect of these results is the insight they provide into the relative difficulty of the problems. The success rates for Symbolic Integration, Santa Fe and Blocks would suggest a high density of solutions within the search space for these problems.

A more detailed examination of Random Search in the case of Symbolic Integration and the Santa Fe Trail indicates that there is no revisiting of previously encountered individuals, that is each of the 25000 samples are unique. This lack of duplication is maintained even when one reduces the maximum permitted value of the codon to the minimum possible, that is the number of productions in the grammar (i.e. removal of Genetic Code Degeneracy see Section~\ref{degeneracy}). For example fixing the maximum codon value to 2 in the case of Santa Fe and 3 in the case of Symbolic Integration.

Another emerging feature from these results is the correlation between average number of expressed codons in a solution and success rates. Symbolic Integration proves to be the easiest of the problems to solve, requiring on average just 14 codons (see Table~\ref{rs_results_analysis_table}) while Santa Fe and Blocks require on average 46 and 69 codons respectively.

\section{Characteristics of Solutions found by Random Search}
Table~\ref{rs_results_analysis_table} shows some of the characteristics of the solutions found by Random Search. Symbolic Integration requires on average only 14 codons, while Santa Fe and Blocks require 46 and 69 respectively. Wrapping occurs in 43\% of Santa Fe solutions and 41\% of Blocks solutions while none of the Symbolic Integration solutions use wrapping. Wrapping on the Blocks problem often results in multiple wrap events on the same genome, averaging two wrap events per succesful solution. Santa Fe averages just one wrap event for successfull solutions.

\begin{table}[h]
\begin{center}
\begin{tabular}{|l|l|l|l|l|}
\hline
Feature & Sym Int & Santa Fe & Blocks & Sym Reg  \\
\hline
Avg Number of Codons & & & &  \\ 
in Solution & 57 & 57 & 58 & n/a  \\
Avg Number of expressed & & & &  \\
Codons in Solution & 14 & 46 & 69 & n/a  \\
Percentage of Solutions & & & &  \\
featuring Wrapping & 0\% & 43\%  & 41\% & n/a  \\
\hline
\end{tabular}
\caption{\label{rs_results_analysis_table} Analysis of Characteristics from Solutions found by Random Search.}
\end{center}
\end{table}



\section{Impact of wrapping}
Wrapping appears prominently in the Santa Fe and Blocks problems with 43\% of the Santa Fe solutions and 41\% of Blocks solutions featuring wrapping.

 Symbolic Integration doesn't use wrapping in any of the solutions found by Random Search. An examination of the average number of expressed codons used in solutions to the Symbolic Integration problem shows it to be quite short, at 14, which contrasts with the figure of 57 for the average number of codons used in a successful genome. This ratio of actual codons provided to actual codons required is greatest for Symbolic Integration. Comparing this same ratio for Santa Fe and Blocks reveals that although both problems use wrapping at similar levels (41\% - 43\%) the ratio for Blocks is actually less than one indicating a much higher number of wrap events. An analysis of the Blocks solutions supports this, revealing up to four wrap events within a single solution on occasion. 

Column 2 of Table~\ref{rs_no_wrap_results_table} shows the impact on success rates of removing wrapping. Symbolic Integration which showed no dependency on wrapping remains unchanged while Santa Fe and Blocks drop by 15\% and 7\% respectively. 
 

\begin{table}[h]
\begin{center}
\begin{tabular}{|l|l|l|l|l|}
\hline
Problem & Successful & Change in  & Avg Number & Avg Number \\
        & Runs       & Success Rate & of Codons & Codons Used \\
\hline
Symbolic Int. & 41\% & +1\% & 53 & 15 \\
Santa Fe Trail & 15\% & -15\% & 63 & 37 \\
Blocks & 17\% & -7\% & 65 & 39 \\
Symbolic Reg. & 0\% & n/a & n/a & n/a \\
\hline
\end{tabular}
\caption{\label{rs_no_wrap_results_table} Results from 1000 trials of Random Search with Wrapping disabled on the problem set.}
\end{center}
\end{table}


\section{Summary}
In this chapter we have looked at the efforts of Random Search in solving the problem set. Symbolic Integration, Santa Fe Trail and Blocks were solved with surprisingly high success rates, while all trials on Symbolic Regression and Spirals were unsuccessful. Wrapping featured prominently in the results for Santa Fe and Blocks, removing wrapping sees a drop in the success rate for these two problems. If we regard Random Search as a means of providing insight into the density of solutions in the search space then we can conclude that Symbolic Integration requiring an average of 14 codons has the highest density of solutions making it the easiest of the selected problems. Santa Fe and Blocks also show high solution densities requiring an average of 46 and 69 codons respectively. The results suggest that Symbolic Regression is a much more difficult problem, remaining unsolved by random search.





























