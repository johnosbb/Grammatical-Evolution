\chapter{Evolutionary Programming}
\section{Introduction}
Evolutionary Programming is the last of the selected metaheuristics to be evaluated in these trials. Although similar in many respects to Evolutionary Strategies it evolved independently from the work of Fogel, Owens and Walsh in the 1960s. The main difference between the two approaches is Evolutionary Strategies' absence of a recombination operator and its use of a tournament based selection method where an individuals probability of selection is based the number of wins achieved against 10 randomly selected opponents~\cite{back}. Like Evolutionary Strategies it uses a fixed length representation.

We look first at the search strategy options and the experimental set-up used in the trials. Some initial runs are performed to assess the impact of the selected strategy variables before the final results are presented. The chapter concludes by looking at some of the characteristics of solutions found by Evolutionary Programming. 


\section{Search Strategy Options}
A population size of 50 individuals evolving over 500 generations provides the 25000 evaluations of the objective function  used in previous trials. The genome length is fixed at 100 for Symbolic Integration, Santa Fe and Blocks, while a longer length of 200 was used for Symbolic Regression.

The key strategy choice for this method revolves around the selection of suitable strategy variables (see Section~\ref{strategy_variables}). Two have been chosen for these trials, \emph{mutation rate}, which determines the probability of a codon being mutated, and \emph{mutation bias} which determines where in the genome the mutation can take place.  


\section{Experimental Conditions}
Table~\ref{ep_param_table} shows the main parameters used to configure Evolutionary Programming. The initial mutation rate has been set at 12\% based on the results from this parameter value in the Evolutionary Streategies trials. The \emph{mutation rate range} value is set at 25\% (Section~\ref{es_options} provides more detail on these parameters). 

\begin{table}[h]
\begin{center}
\begin{tabular}{|l|l|l|l|l|}
\hline
Parameter &\multicolumn{4}{l|}{Problems}\\
\cline{2-1} \cline{3-1} \cline{4-1} \cline{5-1} 
 & Sym Int & Santa Fe & Blocks & Sym Reg \\
\hline
Number of Trials & 1000 & 1000 & 1000 & 1000  \\
Number of Objective & & & & \\ 
Function Evaluations  & 25000 & 25000 & 25000 & 25000  \\
Fixed Genome Length  & 100 & 100 & 100 & 200 \\
Population Size  & 50 & 50 & 50 & 50  \\
Number of Generations  & 500 & 500 & 500 & 500 \\
Initial Mutation Rate & 12\% & 12\% & 12\% & 12\% \\
Mutation Range & 25\% & 25\% & 25\% & 25\%  \\ 
\hline
\end{tabular}
\caption{\label{ep_param_table} Parameters used to Configure Evolutionary Programming.}
\end{center}
\end{table}


\section{Impact of Strategy Variables}

A number of preliminary runs were performed to evaluate the impact of the selected strategy variables. The first set uses mutation with no positional bias, that is the mutation can fall anywhere within the entire genome length. The second set uses a strategy variable that limits the mutation of codons to a certain point in the genome. It is expressed as a percentage, where for example 0\% means that mutation can take place anywhere in the gemone, 50\% means that mutation is limited to the second half of the genome and 100\% effectively stops any mutation from taking place. 

The results show in Table~\ref{ep_strategies} shows a significant performance benefit for Symbolic Integration, Santafe, Blocks and Symbolic Regression when mutation without bias is used.

\begin{table}[h]
\begin{center}
\begin{tabular}{|l|l|l|}
\hline
&\multicolumn{2}{|l|}{Strategy Variables}\\
\hline
Problem       &  Mutation      &    Mutation   \\
              &  without Bias  &    with Bias  \\
\hline
Symbolic Integration & 94\%   & 42\% \\
Santa Fe Trail       & 73\% & 21\%   \\
Blocks               & 95\% & 64\%   \\
Symbolic Regression  & 25\% & 11\%  \\
\hline
\end{tabular}
\caption{\label{ep_strategies} Analysis of EP Strategy Variables showing Success Rates with and without Mutation Bias}
\end{center}
\end{table}


\section{Results}
Table~\ref{ep_results_table} summarises the overall results for Evolutionary Programming. These final trials use mutation without positional bias, a mutation rate of 12 and a mutation range of 25. Symbolic Regression is solved with the same success rate of Evolutionary Strategies (25\%). Santa Fe is the one problem which sees Evolutionary Programming (73\%) outperform Evolutionary Strategies (63\%). Scores for the Blocks problem showing no significant difference at around 95\%. There is also no significant difference in performance for the Symbolic Integration problem with Evolutionary Strategies scoring 95\% and Evolutionary Programming scoring 94\%.

\begin{table}[h]
\begin{center}
\begin{tabular}{|l|l|}
\hline
Problem & Successful Runs \\
\hline
Symbolic Integration & 94\% \\
Santa Fe Trail & 73\% \\
Blocks & 95\% \\
Symbolic Regression & 25\% \\
\hline
\end{tabular}
\caption{ \label{ep_results_table} Results from Evolutionary Programming Trials.}
\end{center}
\end{table}






\section{Characteristics of Solutions found by Evolutionary Programming}

Table~\ref{ep_results_analysis_table} provides details of the solutions found by Evolutionary Programming A surprising aspect of these results is the emergence of wrapping in some of the solutions. This is curious when one considers that fixed length genome of length 100 are being used in the case of Santa Fe and Blocks. This length is much longer than the number of codons required to solve the respective problems, however Evolutionary Programming still manages to employ wrapping by finding solutions whose number of expressed codons are in excess of 100. This is contrast to the situation with Evolutionary Strategies in the last chapter which also employed fixed length genomes of similar lengths, yet no wrapping was used in any of the solutions found by that method. 

One contributing factor towards the re-emergence of wrapping is that the solutions found by Evolutionary Programming tend to have larger numbers of expressed codons than any of the previous metaheuristics. Santa Fe, for example, uses on average 74 expressed codons as against an average of 46 for all of the previous metaheuristics. An examination of successful solutions shows very high levels of nesting of conditionals in the Santa Fe and Blocks problems. Perhaps the lack of a recombination operator forces Evolutionary Programming toward longer genome lengths in a search for successful solutions.

\begin{table}[h]
\begin{center}
\begin{tabular}{|l|l|l|l|l|}
\hline
Feature & Sym Int & Santa Fe & Blocks & Sym Reg  \\
\hline
Avg Number of Codons & & & &  \\ 
in Solution & 100 & 100 & 100 & 200  \\
Avg Number of expressed & & & &  \\
Codons in Solution & 19 & 74 & 48 & 57 \\
Percentage of Solutions & & & &  \\
using Wrapping & 0\% & 2.79\% & 46\% & 1\% \\
\hline
\end{tabular}
\caption{\label{ep_results_analysis_table} Analysis of Features from Solutions found by Evolutionary Programming.}
\end{center}
\end{table}



\section{Summary}
This chapter looked at the last of our selected metaheuristics, Evolutionary Programming. Despite the absence of a recombination operator, the algorithm successfully solved all four problems. Success rates across all of the problems are not significantly different than the scores achieved by Evolutionary Strategies. 

A significant aspect of Evolutionary Programming solutions was their length, with the number of expressed codons exceeding those of the other metaheuristics, causing two of the problems to employ wrapping even when fixed length genomes of 100 codons were used.











