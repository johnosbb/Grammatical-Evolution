 \begin{thebibliography}{99}   \bibitem{goldberg} Goldberg, David E. 1989. Genetic Algorithms in Search, Optimization and Machine Learning. Addison Wesley.  \bibitem{kirkpatrick} Kirkpatrick, S., Gerlatt, C. D. Jr., and Vecchi, M.P., Optimization by Simulated Annealing, In Science Volume 220, no. 4598, pages 671-680, 1983.   \bibitem{mitchell} Mitchell, M. and Holland, J. H. 1993. When will a Genetic Algorithm Outperform Hill Climbing? In Cowan J., Tesauro G., and Alspector J., (editors): \emph{Advances in Neural Information Processing Systems}, pages 51-58, San Francisco, CA., 1994. Morgan Kauffman.  \bibitem{koza} Koza, J. 1992. {\em Genetic Programming}. MIT Press.  \bibitem{fogel} Chellapilla, K.,Fogel, D.B., 1997. Exploring  Self Adaptive Methods to Improve the Efficiency of Generating Approximate Solutions to Traveling Salesman Problems Using Evolutionary Programming. In Peter J. Angeline, Robert G. Reynolds, John R. McDonnell, Russell C. Eberhart (editors): \emph{ Lecture Notes in Computer Science 1213, Evolutionary Programming VI, 6th International Conference}, pages 361-372. EP97, Indianapolis, Indiana, USA, April 13-16, 1997, Springer 1997, ISBN 3-540-62788-X   \bibitem{back} B\"ack, T. 1996. {\em Evolutionary Algorithms in Theory and Practice}. Oxford University Press.  \bibitem{rosenbluth} Metropolis, N., Rosenbluth A.W., Rosenbluth M.N., Teller A.H. and Teller E., 1958.  Equations of State Calculations by Fast Computing Machines, In Journal of Chemical Physics, vol. 21, pages 1087-1091, 1958.  \bibitem{bill} Langdon W.B. 1999. Size Fair and Homologous Tree Genetic Programming Crossovers. In In W. Banzhaf, J. Daida, A. E. Eiben, M. H. Garzon, V. Honavar, M. Jakiela, and R. E. Smith, (editors): \emph{Proceedings of the Genetic and Evolutionary Computation Conference}, volume 2, pages 1092-1097, Orlando, Florida, USA,13-17 July 1999. Morgan Kaufmann.  \bibitem{ieee2001} O'Neill M., Ryan C. Grammatical Evolution. {\em IEEE Trans. Evolutionary Computation}, Vol. 5 No. 4, August 2001   \bibitem{langdon} Langdon W.B., Poli R. Why Ants are Hard. In Koza, John R., Banzhaf, Wolfgang, Chellapilla, Kumar, Deb, Kalyanmoym Dorigo, Marco, Fogel, David B., Garzon, Max H., Goldberg, David E., Iba, Hitoshi, and Riolo, Rick L. (editors): \emph{Genetic Programming 1998: Proceedings of the Third Annual Conference}, pages 193-201, July 22-25, 1998, University of Wisconsin, Madison, Wisconsin. San Francisco, CA. Morgan Kaufmann.    \bibitem{keijzer} Keijzer M, O'Neill M., Ryan C., and Cattolico, M. (2002) Grammatical Evolution Rules: The Mod and the Bucket Rule. In Foster J.A., Lutton E., Miller J.F., Ryan C., Tettamanzi A., (editors): \emph{Lecture Notes in Computer Science 2278, Proceedings of the Fifth European Conference on Genetic Programming}, pages 123-130, Kinsale, Ireland, April 2002. Springer-Verlag.  \bibitem{oneill} O'Neill M., Ryan C., Keijzer M and Cattolico M. (2001)  Crossover in Grammatical Evolution: The Search Continues. In Miller J., Tomassini M., Lanzi P.L, Ryan C., Andrea G.B., Langdon W.B., (editors): \emph{Lecture Notes in Computer Science 2037, Proceedings of the 4th European Conference on Genetic Programming, EuroGP 2001}, pages 337-347. Lake Como (Milan), Italy, 18-20 April 2001. Springer-Verlag.  \bibitem{oneill2} O'Neill M., Ryan C. (2000)  Crossover in Grammatical Evolution: A Smooth Operator? In Banzhaf W., Fogarty T., Langdon W.B., Miller J., Nordin P., Poli R., (editors): \emph{Proceedings of the Third European Workshop on Genetic Programming 2000}, pages 149-162, Edinburgh, Scotland, UK, April 2000  \bibitem{oneill3} ,O'Neill M., Ryan C. (1999)  Evolving Multi-Line Compilable C Code. In Poli, R., Nordin, P., Langdon, W. B., Fogarty, T. C., (editors): \emph{In Proceedings of the Second European Workshop on Genetic Programming 1999}, pages 83, G�teborg, Sweden, May 26-27, 1999.  \bibitem{oneill4} ,O'Neill M., Brabazon T., Ryan C., Collins J.J  (2001)  Developing a Market Timing System using Grammatical Evolution. In Beyer H., Cantu-Paz E., Goldberg D., Spector L., Whitley D., (editors): \emph{Proceedings of GECCO 2001}, Morgan Kaufmann 2001 ISBN 1558607749    \bibitem{oneill5} ,O'Neill M., Brabazon T., Ryan C., Collins J.J  (2001). Evolving Market Index Trading Rules using Grammatical Evolution. In Boers E.J.W., Cagnoni S., Gottlieb J., Hart E., Lanzi P.L., Raidl G.R., Smith R.E., Tijink H., (editors): In \emph{Lecture Notes in Computer Science 2037, Applications of Evolutionary Computing}, pages 343-35, Springer-Verlag.   \bibitem{ryan1} O'Neill M., Ryan C. (1999)  Genetic Code Degeneracy: Implications for Grammatical Evolution and Beyond. In Floreano D, Nicoud J.D , Mondada F., (editors): \emph{Lecture Notes in Computer Science 1674, Proceedings of the Fifth European Conference on Artifical Life, ECAL99}, Lausanne, Switzerland, September 13-17, 1999. Springer-Verlag.   \bibitem{duvivier} Duvivier, D., Preux, P., Talbi, E.-G. 1996. Climbing up NP-hard hills In Parallel Problem Solving from Nature IV. D. Duvivier, Ph. Preux, E-G. Talbi. Climbing-Up NP-Hard Hills, Proc.Parallel Problem Solving from Nature IV, Berlin, Sep.1996, Springer-Verlag, Lecture Notes in Computer Science.
Michael Sampels
In: R. Wyrzykowski, J. Dongarra, M. Paprzycki, J. Wasniewski (Eds.): Proceedings of the 4th International Conference on Parallel Processing & Applied Mathematics (PPAM '01), Lecture Notes in Computer Science 2328 <http://link.springer.de/link/service/series/0558/tocs/t2328.htm>, pp. 35-41, Springer-Verlag, 2002
ISSN 0302-9743
ISBN 3-540-43792-4

 THIS ONE  \bibitem{carpenter} Carpenter, G., Grossberg, S., Markuzon, N., Reynolds, J., and Rosen, D. (1992). Fuzzy artmap: A neural network architecture for incremental supervised learning of analog multidimensional maps. IEEE Transactions on Neural Networks, 3:698-713. 
G. A. Carpenter, S. Grossberg, N. Markuzon, J. H. Reynolds, and D. B. Rosen, �Fuzzy ARTMAP: A neural network architecture for incremental supervised learning of analog multidimensional maps,� IEEE Trans. Neural Netw., vol. 3, pp. 698�713, Sept. 1992. 

 THIS ONE  \bibitem{fahlman} Fahlman, S. E. and Lebiere, C. (1990). The cascade-correlation learning architecture. In Touretzky (ed.), Advances in Neural Information Processing Systems 2. Morgan Kaufmann.  THIS ONE  \bibitem{lang} Lang, K. J. and Witbrock, M. J. (1988). Learning to tell two spirals apart. In Touretzk D.S., Hinton G.E., Sejnowski T.J., (editors): \emph{Proceedings of the 1988 Connectionists Model Summer School}, pages 52-59.  Morgan Kaufmann.  \bibitem{ripple} Keijzer, M., Ryan, C., O'Neill, M., Cattolico, M. and Vladan, B. (2002). Ripple Crossover In Miller J., Tomassini M., Lanzi P.L, Ryan C., Andrea G.B., Langdon W.B., (editors): \emph{Lecture Notes in Computer Science 2037, Proceedings of the 4th European Conference on Genetic Programming, EuroGP 2001}, pages 74-86. Lake Como (Milan), Italy, 18-20 April 2001. Springer-Verlag. SHOULD THIS BE 2038?????????????????? Genetic Programming. LNCS 2038, Proc. of the Fourth European Conference on Genetic Programming, Lake Como, Italy, April 2001, pp.74-86. Springer.  \bibitem{elseth} Eleseth, G.D, Baumgardner, (1995). Principles of modern genetics. West Publishing Company.  \bibitem{kimura} Kimura, M., (1983). The Neutral Theory of Molecular Evolution. Cambridge University Press.  \bibitem{tabu} Glover, G., Laguna, M. (1995). Tabu Search. In Modern Heuristic Techniques for Combinational Problems, chapter 3. McGraw-Hill Book Company, Berkshire, 1995.   \end{thebibliography}         




