
\begin{thebibliography}{99} 

\bibitem{back} B\"ack T. (1996). {\em Evolutionary Algorithms in Theory and Practice}. Oxford University Press.

\bibitem{carpenter} Carpenter, G., Grossberg, S., Markuzon, N., Reynolds, J., and Rosen, D. (1992). Fuzzy artmap: A neural network architecture for incremental supervised learning of analog multidimensional maps. in \emph{IEEE Trans. Neural Networks., Volume 3}, pages 698-713, Sept. 1992. 


\bibitem{fogel} Chellapilla, K., Fogel, D. B. (1997). Exploring  Self Adaptive Methods to Improve the Efficiency of Generating Approximate Solutions to Traveling Salesman Problems Using Evolutionary Programming. In Angeline, P. J., Reynolds, R. G., McDonnell, J. R., Eberhart, R. C., (editors): \emph{ Lecture Notes in Computer Science 1213, Evolutionary Programming VI, 6th International Conference}, pages 361-372. EP97, Indianapolis, Indiana, USA, April 13-16, 1997, Springer 1997, ISBN 3-540-62788-X 


\bibitem{duvivier} Duvivier, D., Preux, P., Talbi, E.-G. (1996). Climbing up NP-hard hills. In Wyrzykowski, R., Dongarra, J., Paprzycki, M., Wasniewski, J., (editors): \emph{Lecture Notes in Computer Science 2328, Proceedings of the 4th International Conference on Parallel Processing & Applied Mathematics (PPAM '01)}, pages 35-41, Springer-Verlag, 2002 ISBN 3-540-43792-4

\bibitem{elseth} Eleseth, G.D, Baumgardner, K. D. (1995). \emph{Principles of Modern Genetics.} West Publishing Company.

\bibitem{fahlman} Fahlman, S. E. and Lebiere, C. (1990). The cascade-correlation learning architecture. In Touretzky (ed.), Advances in Neural Information
Processing Systems 2. Morgan Kaufmann. 

\bibitem{tabu} Glover, G., Lagunas, M. (1995). \emph{Tabu Search. In Modern Heuristic Techniques for Combinational Problems}, chapter 3. McGraw-Hill Book Company, Berkshire, 1995. 


\bibitem{goldberg} Goldberg, David E. (1989). \emph{Genetic Algorithms in Search, Optimization and Machine Learning.} Addison Wesley.


\bibitem{holland} Holland, J.H. (1992). \emph{Adaptation in Natural and Artifical Systems.} MIT Press.

\bibitem{ripple} Keijzer, M., Ryan, C., O'Neill, M., Cattolico, M. and Vladan, B. (2002). Ripple Crossover In Miller, J., Tomassini, M., Lanzi, P.L, Ryan, C., Andrea, G.B., Langdon, W.B., (editors): \emph{Lecture Notes in Computer Science 2038, Proceedings of the 4th European Conference on Genetic Programming, EuroGP 2001}, pages 74-86. Lake Como (Milan), Italy, 18-20 April 2001. Springer-Verlag.


\bibitem{keijzer} Keijzer, M, O'Neill, M., Ryan, C., and Cattolico, M. (2002). Grammatical Evolution Rules: The Mod and the Bucket Rule. In Foster, J.A., Lutton, E., Miller, J.F., Ryan, C., Tettamanzi, A., (editors): \emph{Lecture Notes in Computer Science 2278, Proceedings of the Fifth European Conference on Genetic Programming}, pages 123-130, Kinsale, Ireland, April 2002. Springer-Verlag.

\bibitem{kimura} Kimura, M. (1983). \emph{The Neutral Theory of Molecular Evolution.} Cambridge University Press.

\bibitem{kirkpatrick} Kirkpatrick, S., Gerlatt, C. D. Jr., and Vecchi, M.P. (1983). Optimization by Simulated Annealing, In \emph{Science Volume 220, No. 4598}, pages 671-680, 1983. 


\bibitem{koza} Koza, J. (1992). \emph{Genetic Programming}. MIT Press.


\bibitem{koza1} Koza, J. (1994). \emph{Genetic Programming II: Automatic Discovery of Reusable Program.} MIT Press.

\bibitem{koza2} Koza, J., Keane, M.A., Streeter, M.J., Mydlowec, W., Yu, J., Lanza, G. (2003). \emph{Genetic Programming IV: Routine Human-Competitive Machine Intelligence.} Kluwer Academic Publishers.

\bibitem{lang} Lang, K. J. and Witbrock, M. J. (1988). Learning to tell two spirals apart. In Touretzk, D.S., Hinton, G.E., Sejnowski, T.J., (editors): \emph{Proceedings of the 1988 Connectionists Model Summer School}, pages 52-59.  Morgan Kaufmann.

\bibitem{langdon} Langdon, W. B., Poli R. (1998). Why Ants are Hard. In Koza, J. R., Banzhaf, Wolfgang, Chellapilla, Kumar, Deb, Kalyanmoym Dorigo, Marco, Fogel, D. B., Garzon, Max, H., Goldberg, D. E., Iba, Hitoshi, and Riolo, Rick L. (editors): \emph{Genetic Programming 1998: Proceedings of the Third Annual Conference}, pages 193-201, July 22-25, 1998, University of Wisconsin, Madison, Wisconsin. San Francisco, CA. Morgan Kaufmann. 

\bibitem{bill} Langdon, W. B. (1999). Size Fair and Homologous Tree Genetic Programming Crossovers. In In Banzhaf, W., Daida, J., Eiben, A. E., Garzon, M. H., Honavar, V., Jakiela, M., and Smith, R. E., (editors): \emph{Proceedings of the Genetic and Evolutionary Computation Conference}, Volume 2, pages 1092-1097, Orlando, Florida, USA,13-17 July 1999. Morgan Kaufmann.


\bibitem{rosenbluth} Metropolis, N., Rosenbluth, A. W., Rosenbluth, M. N., Teller, A. H. and Teller, E. (1958).  Equations of State Calculations by Fast Computing Machines, In \emph{Journal of Chemical Physics, Volume 21}, pages 1087-1091, 1958.

\bibitem{mitchell} Mitchell, M. and Holland, J. H. (1993). When will a Genetic Algorithm Outperform Hill Climbing? In Cowan, J., Tesauro, G., and Alspector, J., (editors): \emph{Advances in Neural Information Processing Systems}, pages 51-58, San Francisco, CA., 1994. Morgan Kauffman.

\bibitem{ieee2001} O'Neill, M., Ryan, C. (2001). Grammatical Evolution. {\em IEEE Trans. Evolutionary Computation}, Volume 5 No. 4, August 2001 



\bibitem{oneill} O'Neill, M., Ryan, C., Keijzer, M and Cattolico, M. (2001).  Crossover in Grammatical Evolution: The Search Continues. In Miller, J., Tomassini, M., Lanzi, P. L., Ryan, C., Andrea, G. B., Langdon, W. B., (editors): \emph{Lecture Notes in Computer Science 2038, Proceedings of the 4th European Conference on Genetic Programming, EuroGP 2001}, pages 337-347. Lake Como (Milan), Italy, 18-20 April 2001. Springer-Verlag.


\bibitem{oneill2} O'Neill, M., Ryan, C. (2000).  Crossover in Grammatical Evolution: A Smooth Operator? In Banzhaf, W., Fogarty, T., Langdon, W. B., Miller, J., Nordin, P., Poli, R., (editors): \emph{Proceedings of the Third European Workshop on Genetic Programming 2000}, pages 149-162, Edinburgh, Scotland, UK, April 2000

\bibitem{oneill3} O'Neill, M., Ryan, C. (1999).  Evolving Multi-Line Compilable C Code. In Poli, R., Nordin, P., Langdon, W. B., Fogarty, T. C., (editors): \emph{In Proceedings of the Second European Workshop on Genetic Programming 1999}, pages 83, G�teborg, Sweden, May 26-27, 1999.

\bibitem{oneill4} O'Neill, M., Brabazon, T., Ryan, C., Collins, J.J  (2001).  Developing a Market Timing System using Grammatical Evolution. In Beyer, H., Cantu-Paz, E., Goldberg, D., Spector, L., Whitley, D., (editors): \emph{Proceedings of GECCO 2001}, Morgan Kaufmann 2001 ISBN 1558607749 


\bibitem{oneill5} O'Neill, M., Brabazon, T., Ryan, C., Collins, J.J  (2001). Evolving Market Index Trading Rules using Grammatical Evolution. In Boers, E.J.W., Cagnoni, S., Gottlieb, J., Hart, E., Lanzi, P.L., Raidl, G.R., Smith, R.E., Tijink, H., (editors): In \emph{Lecture Notes in Computer Science 2037, Applications of Evolutionary Computing}, pages 343-35, Springer-Verlag. 


\bibitem{oneill6} O'Neill, M., Keijzer, M. and Cattolico, M., Ryan, C.  (2003). Crossover in Grammatical Evolution. In Ryan, C., Soule, T., Keijzer, M., Tsang, E., Poli, R., Costa, E. (editors): In \emph{Lecture Notes in Computer Science 2610, Proceedings of the Sixth European Conference, EuroGP 2003}, Springer-Verlag.

\bibitem{ryan1} O'Neill, M., Ryan, C. (1999).  Genetic Code Degeneracy: Implications for Grammatical Evolution and Beyond. In Floreano, D, Nicoud, J.D , Mondada, F., (editors): \emph{Lecture Notes in Computer Science 1674, Proceedings of the Fifth European Conference on Artifical Life, ECAL99}, Lausanne, Switzerland, September 13-17, 1999. Springer-Verlag. 

\bibitem{mike_thesis} O'Neill, M. (2001). Automatic Programming in an Arbitrary Language: Evolving Programs with Grammatical Evolution. PhD thesis, University of Limerick. M0005921LK 


\bibitem{ge_book} Ryan, C., O'Neill, M. (2003). \emph{Grammatical Evolution. Evolutionary Automatic Programming in an Arbitrary Language.} Kluwer Academic Publishers.



\end{thebibliography}








