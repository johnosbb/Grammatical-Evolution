\chapter{Evolutionary Strategies}
\section{Introduction}
This chapter looks at Evolutionary Strategies, the second of three population based methods. Like Genetic Algorithms, it uses both mutation and recombination operators to explore the search space through successive generations. One notable difference between the Genetic Algorithms and Evolutionary Strategies is the the use of a fixed length representation in the latter. The use of fixed length representations is motivated by a desire to be consistent with the commonly used forms of Evolutionary Strategies and the difficulty associated with using variable-length genomes with the established recombination operators (see Section~\ref{es_recombination}). 

We look initially at the options  used to configure the basic algorithm before examining the principle strategy choices of selection and recombination. The results of the main trials are presented with some commentary and analyses of the solutions found by this method.


\section{Search Strategy Options}
\label{es_options}For Evolutionary Strategies the population size has been set at 50 (this population size is typical of Evolutionary Strategies~\cite{back}), while the number of generations is set to 500, this provides us with 25000 evaluations of the objective function as used in previous trials. The genome length is fixed at 100 for Symbolic Integration, Santa Fe and Blocks. A longer length of 200 was used for Symbolic Regression.  An initial \emph{mutation rate} of 12\% was chosen with a \emph{mutation rate range} of 25\% based on results from preliminary trials. This initial rate of 12\% is then mutated for each individual as it evolves with that individual through each generation. The \emph{mutation rate range} value of 25\% is the upper limit permissible for the mutation rate after the first generation. So, for example, in the first generation an individual could be subjected to between 0\% and 12\% mutation and between 0\% and 25\% on every subsequent generation. A more complete explanation of this process is provided in Section~\ref{es}. The mutation operator operates at the codon level, with the mutation rate expressed as a percentage of the number of codons in a genome, so for example, a figure of 12\% means that up to 12\% of the codons in a genome could be mutated in the operation. 



\section{Experimental Conditions}
Table~\ref{es_param_table} shows the main parameters used to configure Evolutionary Strategies. In order to determine the most suitable selection and recombination methods a number of preliminary trials were performed. These are covered in the next section. 



\begin{table}[h]
\begin{center}
\begin{tabular}{|l|l|l|l|l|}
\hline
Parameter &\multicolumn{4}{l|}{Problems}\\
\cline{2-1} \cline{3-1} \cline{4-1} \cline{5-1} 
 & Sym Int & Santa Fe & Blocks & Sym Reg \\
\hline
Number of Trials & 1000 & 1000 & 1000 & 1000  \\
Number of Objective & & & & \\ 
Function Evaluations  & 25000 & 25000 & 25000 & 25000  \\
Fixed Genome Length  & 100 & 100 & 100 & 200 \\
Population Size  & 50 & 50 & 50 & 50  \\
Number of Generations  & 500 & 500 & 500 & 500  \\
\hline
\end{tabular}
\caption{\label{es_param_table} Parameters used to Configure Evolutionary Strategies.}
\end{center}
\end{table}



\section{Evolutionary Strategies Selection Methods}
As discussed in Section~\ref{es_selection_strategies} there are two preferred methods of selection used in ES, the $(\mu + \lambda)-ES$ (plus) strategy allows the best $\mu$ individuals from the union of parents ($\mu$) and offspring ($\lambda$) to survive while the ($\mu,\lambda$) (comma) strategy allows the best ($\mu$) offspring's survive to the next generation.

Evolutionary Strategies also includes a selection of recombination operators (see Section~\ref{es_recombination}), discrete recombination involves selecting two parents at random from the population and creating an offspring, which consists of object variables genes drawn with equal probability from both parents, while panmictic (global) intermediate recombination involves selecting one parent at random from the population and creating an offspring through recombination with other randomly selected parents.

To determine the best combination of these options an initial trial of 500 runs was performed, the results of which are shown in Table~\ref{es_strategies}. An initial mutation rate of 12\% and a mutation range of 25\% were used for these preliminary trials. The \emph{plus} selection strategy which allows the best individuals from the union of parents and offspring to survive has a distinct advantage while intermediate recombination which involves recombination of an individual with a number of other randomly selected parents has a slight advantage over its discrete counter part. 



\begin{table}[h]
\begin{center}
\begin{tabular}{|l|l|l|l|l|}
\hline
&\multicolumn{4}{|l|}{Selection Strategy}\\
\hline
Problem Type  & Comma         &  Comma     & Plus         & Plus \\
              & Intermediate  &  Discrete  & Intermediate &  Discrete \\
\hline
Symbolic Integration  & 25.2\% & 28.8\% & 95.4\% & 95.4\% \\
Santa Fe Trail & 17.2\% & 24.2\% & 54.2\% & 51\% \\
Blocks & 25.4\% & 23.8\% & 99.2\% & 98.4\% \\
Symbolic Regression  & 0.2\% & 0\% & 25.2\% & 20.2\% \\
\hline
\end{tabular}
\caption{\label{es_strategies} Comparative Analysis of ES Strategies.}
\end{center}
\end{table}



\section{Impact of Mutation Rate on Success}
One of the other key strategy choices with Evolutionary Strategies is the rate of mutation. Two factors influence this in these trials, the initial mutation rate is the starting rate for the individuals in the initial population, the mutation rate range is an upper limit imposed on the mutation rate as it evolves and adapts with the individual. 

To evaluate the impact of different values for these parameters a number of trials were conducted on two of the problems, Santa Fe and Symbolic Regression. Table~\ref{es_combined_mutation_rate} shows the results of trials on the Santa Fe and Symbolic Regression problem. The Santa Fe problems shows significant improvement for higher rates of mutation, while on the Symbolic Regression problem there is a statistically significant difference for mutation ranges less than 25\% indicating that there is a minimum level of mutation that is required to maintain a reasonable success rate.
 
\begin{table}[h]
\begin{center}
\begin{tabular}{|l|l|l|l|l|}
\hline
Initial Rate  &  Range & \multicolumn{2}{l|}{\% Success}  \\
\hline
              &        & Santa Fe & Sym Reg \\
\hline
5\% & 5\% & 13\%      & 13\% \\
5\% & 10\% & 33\%     & 29\% \\
5\% & 25\% & 54\%     & 21\% \\
12\% & 25\% & 54.2\%  & 25.2\%\\
15\% & 25\% & 63\%    & 22\%\\
15\% & 40\% & 66\%    & 25\%\\
25\% & 25\% & 50\%    & 22\ \\
\hline
\end{tabular}
\caption{\label{es_combined_mutation_rate} Comparative Analysis of Mutation Rates and Ranges for Santa Fe and Symbolic Regression}
\end{center}
\end{table}



\section{Results}

The final results for the Evolutionary Strategy trials are provided in Table~\ref{es_results_table}. All of these trials use intermediate recombination with a \emph{plus} selection strategy. An initial mutation rate of 15\% and a mutation rate of 25\% have been used.

Symbolic Integration and Blocks have the highest success rates at 95\% and 96\% respectively. Santa Fe is solved in 63\% of the trials and Symbolic Regression has a success rate of 22\%. This is only the second metaheuristic to solve the Symbolic Regression problem with Genetic Algorithms previously solving it with a success rate of 36\%. 


\begin{table}[h]
\begin{center}
\begin{tabular}{|l|l|}
\hline
Problem & Successful Runs \\
\hline
Symbolic Integration & 95\% \\
Santa Fe Trail & 63\% \\
Blocks & 96\% \\
Symbolic Regression & 25\% \\
\hline
\end{tabular}
\caption{ \label{es_results_table} Final Results from Evolutionary Strategies Trials.}
\end{center}
\end{table}




\section{Characteristics of Solutions found by Evolutionary Strategies}
Table~\ref{es_results_analysis_table} shows the characteristics of the solutions found by Evolutionary Strategies. Because we are using fixed lengths all of the solutions have a length of 100 or more codons ( 200 in the case of Symbolic Regression), this also helps to account for the figure of 0\% for wrapping because the required number of codons for a solution is always less than 100 for these particular problems. 

When ES was forced to use wrapping by restricting the maximum genome length to be less than the average solution length it failed to find any solutions for the Symbolic Integration problem. A similar experiment on Santa Fe saw a success rate of 55\% even when genome lengths were restricted.

The numbers of expressed codons in a successful solution are largely in line with previous results, again the Blocks problem requires the largest number of codons at 53 with Symbolic Regression requiring 49 on average.


\begin{table}[h]
\begin{center}
\begin{tabular}{|l|l|l|l|l|}
\hline
Feature & Sym Int & Santa Fe & Blocks & Sym Reg \\
\hline
Avg Number of Codons & & & &  \\ 
in Solution & 100 & 100 & 100 & 100 \\
Avg Number of expressed & & & &  \\
Codons in Solution & 18 & 53 & 39 & 49 \\
Percentage of Solutions & & & &  \\
featuring Wrapping & 0\% & 0\% & 0\% & 0\%  \\
\hline
\end{tabular}
\caption{\label{es_results_analysis_table} Analysis of Characteristics from Solutions found by Evolutionary Strategies.}
\end{center}
\end{table}




\section{Summary}
In this chapter we looked at the performance of Evolutionary Strategies, a population based method similar in many respects to Genetic Algorithms. We examined a number of search strategy choices, including selection strategy, recombination operator and mutation rates. We found a \emph{plus} selection strategy and a intermediate recombination operator to be the preferred choice for all of the problems. We also looked at the impact of the mutation rate and found the higher levels of mutation (25\%) were more successful on the two problems we selected. Evolutionary strategies has proved itself successful on all four problems, although with lower numbers of successful trials than Genetic Algorithms. 










