\chapter{Introduction to the Thesis}
\section{Introduction}

Grammatical Evolution (GE) is an Evolutionary Algorithm (EA) that uses variable length binary strings to select productions from a Backus Naur Form grammar. A feature of Grammatical Evolution is the ability to separate the search space from the solution space. This aspect, which allows different methods to be used to explore the search space, is examined in this research by evaluating the performance and characteristics of GE using a diverse range of search methods (Metaheuristics) on a selection of benchmark problems from the field of \emph{Evolutionary Computation (EC)}.


\section{Contributions of the Thesis}
The principal contribution  of the thesis is the evaluation of GE with a range of metaheuristics. In an effort to better understand the nature of the search space presented by the GE process these metaheuristics have been evaluated on a range of traditional Genetic Programming (GP) benchmark problems.

This is the first time that research on GE has focused on using different search strategies, up to this point GE had been exclusively used with a Variable-Length Genetic Algorithm as the search engine.
 
The work brings together a useful practical summary of many of the key metaheuristics in use today, providing a step-by-step guide to their underlying algorithms and operators. The research has also resulted in the generation of a significant body of object orientated code implementing all of the selected  metaheuristics.
By using Random search as one of the search mechanisms in this research we provide some interesting insights into the relative difficulty of the selected problems.


\section{Organisation of the Thesis}
Chapter two of the thesis provides an introduction to GE, providing an insight into the key workings of the technique. It provides some background on the specification of grammars, introduces the Backus Naur Form of grammar and illustrates the mapping process used by GE in evolving programs from a grammar.

In chapter three we discuss Metaheuristics, providing an overview of the subject before dealing in detail with each of the metaheuristics featured in this research.

The selected problem set consisting of  \emph{The Santa Fe Trail problem}, \emph{Symbolic Integration}  \emph{Symbolic Regression} and \emph{Block Stacking} is introduced in chapter four, where we provide a description of each problem and presents its associated grammar.

Chapters seven through to ten focus on the results from experimental trials using the selected Metaheuristics, each chapter follows a similar structure presenting the overall results and then attempting to analyse and improve the performance of the Metaheuristic under consideration.

The focus of chapter eleven is a comparative  overview of all of the results provided in the previous chapters looking at relative performance with some discussion on the characteristics that the various Metaheuristics have in common and a contrast of those aspects that differentiate them. 

Finally, chapter twelve summarises all of the key findings and identifies areas for future research.


\section{Terminology}
In the context of this thesis the word \emph{Solutions} means a potential solution to a problem rather than the correct solution to the problem. A solution that actually solves the problem will be referred to as a \emph{correct solution} or a \emph{successful solution}.

All results have been tested for significance using a \emph{Chi Square} significance test at a 98\% confidence level. This test compares the difference between the actual frequencies (the observed frequencies in the data) with those which one would expect if no factor other than chance had been operating (the expected frequencies). The closer these two results are to each other, the greater the probablity that the observed frequencies are influenced by chance alone. When the text refers to changes as being significant or not significant it is within the context of this test.  













































